% Failing Food Truck - Teacher Entry Hook Guide
% Algebra 2 - Parabolas for Profit Day 1
\documentclass[11pt]{article}

% Package imports
\usepackage[margin=1in]{geometry}
\usepackage{enumitem}
\usepackage{amsmath}
\usepackage{xcolor}
\usepackage{tcolorbox}
\usepackage{titlesec}

% Formatting
\setlist[itemize]{leftmargin=*, itemsep=2pt, topsep=2pt}
\setlist[enumerate]{leftmargin=*, itemsep=2pt, topsep=2pt}
\titleformat{\section}{\large\bfseries}{}{0pt}{}
\setlength{\parindent}{0pt}

% Document begins
\begin{document}

% Header
\begin{center}
  {\LARGE\textbf{Failing Food Truck Entry Hook}}\\[3pt]
  {\large Teacher Guide - 10 Minutes}
\end{center}

\vspace{0.3em}
\hrule height 0.8pt
\vspace{0.5em}

% The Story with Dialog
\section*{The Story (3-4 minutes)}

\begin{tcolorbox}[colback=blue!5, colframe=blue!50, title={\textbf{Tell This Story (Use Key Terms in Bold)}}]
``I know someone who started a food truck selling gourmet grilled cheese. They were so excited! But after three months, they were losing money and had to quit. Want to know what went wrong?''

\vspace{0.3em}
\textit{[Pause for responses]}

\vspace{0.3em}
``Here's what happened: They set their \textbf{price} at \$15 per sandwich because it was fancy. But at that price, \textbf{demand} was low -- only 20 people bought per day. Their \textbf{revenue} was \$15 $\times$ 20 = \$300 daily.

But here's the problem: Their \textbf{fixed costs} were \$200 per day -- that's truck rental, permits, insurance. These stay the same no matter what. Their \textbf{variable costs} were \$5 per sandwich -- that's ingredients, napkins, containers. So total \textbf{cost} was \$200 + (20 $\times$ \$5) = \$300.

Their \textbf{profit}? Revenue minus cost: \$300 - \$300 = \$0. They hit \textbf{break-even} every day but never made money!

They tried dropping the price to \$8. Now 100 people bought! But wait... Revenue: \$8 $\times$ 100 = \$800. Cost: \$200 + (100 $\times$ \$5) = \$700. Profit: only \$100.

The owner wondered: \textit{`What price would maximize my profit?'}

That's exactly what we're going to figure out mathematically!''
\end{tcolorbox}

% Driving Question
\section*{The Driving Question (1 minute)}

\begin{tcolorbox}[colback=green!5, colframe=green!50]
\textbf{Write on Board:}\\[3pt]
{\large ``How can we use math to find the price that maximizes profit?''}
\end{tcolorbox}

% Board Work
\section*{Write on Board (3-4 minutes)}

\begin{tcolorbox}[colback=yellow!5, colframe=yellow!50, title={\textbf{Quick Discussion Questions}}]
Write these on the board and collect student responses:

\vspace{0.3em}
\textbf{1. Why didn't the \$15 price work?}
\begin{itemize}
  \item Possible answers: ``Too expensive'' / ``Not enough customers'' / ``Low demand''
\end{itemize}

\vspace{0.3em}
\textbf{2. Why wasn't \$8 much better?}
\begin{itemize}
  \item Possible answers: ``Too cheap'' / ``High costs ate the profit'' / ``More work, little gain''
\end{itemize}

\vspace{0.3em}
\textbf{3. What happens to demand when price goes up?}
\begin{itemize}
  \item Possible answers: ``Goes down'' / ``Fewer people buy'' / ``Inverse relationship''
\end{itemize}

\vspace{0.3em}
\textbf{4. What price might work better?}
\begin{itemize}
  \item Possible answers: ``\$10-12'' / ``Somewhere in the middle'' / ``Need to test''
\end{itemize}
\end{tcolorbox}

% Key Terms Summary
\section*{Key Terms to Emphasize (Write on Board)}

\begin{tcolorbox}[colback=gray!5, colframe=gray!50]
\begin{multicols}{2}
\begin{itemize}
  \item \textbf{Revenue} = Price $\times$ Quantity
  \item \textbf{Fixed Cost} = Same every day
  \item \textbf{Variable Cost} = Per item cost
  \item \textbf{Total Cost} = Fixed + Variable
  \item \textbf{Profit} = Revenue - Cost
  \item \textbf{Demand} = How many buy
  \item \textbf{Break-Even} = Profit = 0
  \item \textbf{Price} = What we charge
\end{itemize}
\end{multicols}
\end{tcolorbox}

% Transition
\section*{Transition to Worksheet (1 minute)}

\begin{tcolorbox}[colback=red!5, colframe=red!50]
\textbf{Say:} ``This is our mission for the next few days -- finding that perfect price mathematically. Let's start by making sure we all understand these business terms. Take out your worksheet...''

\vspace{0.3em}
\textbf{Distribute:} Day 1 Launch Worksheet

\vspace{0.3em}
\textbf{Time:} 5-7 minutes for worksheet completion
\end{tcolorbox}

% Teacher Notes
\section*{Quick Teacher Notes}

\begin{itemize}
  \item Keep the story conversational and energetic
  \item Use actual numbers to make it concrete
  \item Don't explain the quadratic connection yet -- that comes Day 2
  \item If students ask ``Why not just try different prices?'' say ``Great question! We'll see how math can find the answer faster than trial and error''
  \item The story sets up the need for optimization without using that word yet
\end{itemize}

\end{document}